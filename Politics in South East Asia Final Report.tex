\documentclass[12pt]{article}
\usepackage[utf8]{inputenc}
\usepackage[english]{babel}
\usepackage{amsmath, amssymb}
\usepackage{hyperref}
\usepackage{geometry}
\usepackage{titlesec}
\usepackage{setspace}
\usepackage{csquotes}
\usepackage{fancyhdr}
\usepackage{enumitem}
\usepackage{cite}

\geometry{margin=1in}
\setlength{\parindent}{0pt}
\setlength{\parskip}{1em}
\onehalfspacing

\title{The Role of Civil Society in the Fall of Authoritarian Regimes: The Philippines (1986) and Indonesia (1998)}
\author{William Evan Smith\\National Chengchi University}
\date{}

\pagestyle{fancy}
\fancyhf{}
\rhead{\thepage}
\lhead{Civil Society and Authoritarianism}

\begin{document}

\maketitle

\begin{abstract}
This study explores the pivotal role of civil society in the democratic transitions of the Philippines (1986) and Indonesia (1998), highlighting how non-institutional actors contributed to the downfall of entrenched authoritarian regimes. Drawing on diverse theoretical frameworks — from Hegel, Marx, and Gramsci to Habermas — it conceptualizes civil society as a contested and dynamic space of resistance, negotiation, and mobilization. In both case studies, civil society actors such as religious institutions, student groups, NGOs, and independent media formed coalitions that leveraged mass mobilization and crises (political or economic) to challenge state authority. While these efforts enabled regime collapse, the study emphasizes that civil society’s role in sustaining democratic consolidation remains fraught. The return of former regime elites — exemplified by Marcos Jr. in the Philippines and Prabowo in Indonesia — reveals the fragility of democratic gains and the enduring influence of old power structures. Ultimately, the analysis underscores that civil society must evolve from an oppositional force into a long-term agent of institutional reform and political accountability.
\end{abstract}


\newpage

\section*{Introduction}
\addcontentsline{toc}{section}{Introduction}
This study focuses on the role of civil society in two major political transitions in Southeast Asia: the People Power Revolution in the Philippines in 1986 and the fall of Suharto’s regime in Indonesia in 1998. These two cases illustrate how non-institutional movements contributed to challenging deeply entrenched authoritarian regimes.
The first part is devoted to a theoretical approach to the concept of civil society. It explores various perspectives, from classical thinkers like Hegel, who sees civil society as an intermediate space between the individual and the state, to Marx, who detects class dynamics within it, and more contemporary analyses that highlight the political role of NGOs, social movements, and community actors under authoritarian systems. Thinkers such as Gramsci, with his concept of cultural hegemony, enrich the reflection on civil society’s capacity to challenge the established order. This theoretical foundation provides the necessary conceptual tools to understand the specific role of civil society in political transition contexts, especially in authoritarian regimes where room for maneuver is limited but where unprecedented alliances can emerge among popular movements, religious actors, independent media, and dissident elites.
The second part analyzes two case studies. In the Philippines, it shows how an unprecedented coalition — the Church, media, students, and NGOs — enabled the largely peaceful overthrow of Ferdinand Marcos, rallying around the figure of Corazon Aquino. In Indonesia, it examines how student mobilizations, catalyzed by the 1997 economic crisis, led to an irreversible challenge to Suharto’s regime. The analysis highlights the internal dynamics of each movement, their mobilization strategies, and the role played by certain segments of power.
Finally, a comparison of both cases will assess the impact of these movements on democratization processes. It will question the achievements of these transitions, the durability of civil society forces, and the obstacles encountered after the fall of the regimes. This study thus seeks to understand how civil society, in authoritarian contexts, can not only catalyze change but also contribute to the construction of a more democratic political order.
How did civil society in the Philippines and Indonesia play a role in bringing down authoritarian regimes?
In this analysis, we will draw on the article Civil Society and Political Change: An Analytical Framework by Muthiah Alagappa, which offers an in-depth review of various theoretical approaches to civil society. This text highlights the diversity of conceptions developed over time, shaped by historical contexts, intellectual traditions, and political frameworks. While the notion of civil society is widely used today, it has not always referred to the same reality, nor served the same purposes.

\section{Theoretical Approaches to Civil Society}

Indeed, according to German philosopher Georg Wilhelm Friedrich Hegel, civil society is an intermediate space between the private sphere of the family and the more universal sphere of the state. For him, it is the place where individuals pursue their particular interests, especially through economic activity. Although it plays an important functional role in social organization, Hegel considers civil society incomplete in itself: left to its own logic, it risks producing social conflict, imbalance, and deep inequality. Only the state, in his view, can provide the cohesion that civil society cannot generate on its own.
A few decades later, Karl Marx offered a radically different interpretation. He saw civil society not as a neutral space or an extension of economic activity but as the central arena of production relations. It is within this sphere that power relations between the bourgeoisie, which owns the means of production, and the proletariat, which owns only its labor power, play out. For Marx, civil society is thus deeply marked by class inequality and the interests of a dominant minority. Where Hegel saw a useful but chaotic space, Marx saw a structure of ideological and economic domination. The two thinkers differ not only in their analysis but also in their vision of civil society’s role in political order.
Antonio Gramsci, a 20th-century thinker, extended Marx’s critique by adding an important nuance. For Gramsci, civil society is not just an economic or institutional space: it is also a cultural battleground, where ideological hegemony is contested. Dominant classes use civil society to disseminate their ideas, values, and norms to stabilize their power without relying constantly on coercion. However, Gramsci also saw civil society as a strategic terrain for emancipatory movements: it can become a site of resistance, where new worldviews are elaborated and spread.
Contemporary theories of civil society take a different turn, particularly after the democratic transitions of the late 20th century. They emphasize civil society’s capacity to foster citizen participation and play an active role in democratizing political regimes. One of the most influential thinkers in this regard is Jürgen Habermas. For him, civil society is embodied in what he calls the public sphere: a symbolic and social space where citizens can freely discuss, debate, and formulate criticism, notably through the media, associations, or social movements. This space for dialogue is an essential counterweight to political and economic power, enabling individuals to collectively shape an enlightened and autonomous public opinion.
These contemporary theories stress the autonomy of civil society. It is not reducible to the state, the market, or even to formal associations: it encompasses all social relationships, forms of engagement, and collective dynamics that allow citizens to act as political agents. In post-authoritarian societies, this autonomy becomes even more crucial, as it allows individuals and groups to pressure institutions, claim their rights, and rebuild a democratic fabric often weakened by years of repression.
Thus, through this diversity of conceptions, it becomes clear that civil society is far from a fixed notion. It is constantly evolving, and its role differs according to historical, political, and cultural contexts. What remains central, however, is the idea that it constitutes an intermediate and dynamic space where interests, ideas, and resistance are expressed.



\section{Case Studies and Comparative Analysis}

The Philippines before 1986 was ruled by President Ferdinand Marcos, who came to power in 1965. His regime was marked by the declaration of martial law in 1972 and the dissolution of democratic institutions. Marcos established an authoritarian regime, backed by the United States in the context of the Cold War, with the Western bloc opposing the communist bloc. His regime was characterized by violent repression of opponents, widespread human rights abuses, and massive corruption at the highest levels of the state. The year 1983 marked a turning point with the assassination of his main exiled opponent, Benigno Aquino Jr., upon his arrival at Manila airport. This event sent shockwaves through the country and reignited popular anger against the regime. Discontent continued to grow, fueled by the economic crisis and widespread rejection of authoritarianism. In response to mounting pressure, Marcos decided to hold a presidential election in 1986, hoping to restore his legitimacy. He faced Corazon Aquino, widow of Benigno Aquino Jr., who had become a symbol of opposition and democracy. Although initial results showed Aquino in the lead, Marcos declared himself the winner in an election marred by massive fraud.This electoral fraud triggered an unprecedented popular mobilization, even involving the powerful Catholic Church, known as the People Power Revolution. Millions of Filipinos peacefully took to the streets of Manila to demand Marcos’s departure. Under popular pressure and with the withdrawal of U.S. support, Marcos was forced to flee the country on February 25, 1986. Corazon Aquino then became president, marking the return of democracy to the Philippines.
Indonesia before 1998 was ruled by President Suharto, who came to power in 1967 after ousting President Sukarno in a military coup. His regime, known as the New Order, emerged during the Cold War and was supported by Western countries, especially the United States, due to its strong anti-communist stance. Suharto established an authoritarian regime based on military control, political repression, and limited civil liberties. During the 1970s and 1980s, Indonesia experienced sustained economic growth, but also rampant corruption, nepotism, and a concentration of power within Suharto’s family and close allies. The economy was dominated by networks of businesses linked to the regime, while human rights abuses were frequent, notably in East Timor. The situation deteriorated significantly in the late 1990s. In 1997, the Asian financial crisis hit Indonesia hard: the currency collapsed, unemployment surged, and poverty worsened. Popular anger grew, fueled by inequality and government corruption. In May 1998, massive protests erupted nationwide, particularly in Jakarta, led by students, unions, and the urban middle class. The violent repression of protests, which resulted in hundreds of deaths, sparked national and international outrage. Facing overwhelming opposition and the withdrawal of military support, Suharto resigned on May 21, 1998, ending over 30 years of authoritarian rule. His vice president, B. J. Habibie, succeeded him and initiated major political reforms, marking the beginning of a democratic transition known as the Reformasi era.
In both cases, in the Philippines and Indonesia, civil society played a central role, though with dynamics specific to each context. In the Philippines, the anti-Marcos movement was largely driven by well-structured networks: the Catholic Church, NGOs, independent media, unions, intellectuals, and student movements. The 1983 assassination of Benigno Aquino Jr. was the catalyst for a movement that had long been simmering. What stands out is civil society’s ability to build a united front, despite repression and fear, and to adopt non-violent forms of protest, such as mass street demonstrations, which proved decisive during the People Power Revolution.
In Indonesia, civil society also became a force for change, but in a more fragmented context, repressed for decades. Only with the 1997–1998 economic crisis did accumulated frustrations erupt. The student movement played a leading role: it was one of the few groups able to occupy public space and demand genuine reform. However, unlike in the Philippines, the Indonesian movement lacked clear leadership or short-term ideological unity, making the democratic transition more uncertain and unstable.
In both cases, civil society was the engine of change, but its influence after regime collapse varied. In the Philippines, although democracy was restored, clientelist dynamics persisted. The Marcos family remained popular, and Ferdinand Marcos Jr. became president in 2022. In Indonesia, although the transition was swift, social expectations were often disappointed by the slow pace of reforms and the survival of old regime elites — most notably with the election of Prabowo Subianto, Suharto’s former son-in-law, as Indonesia’s current president.

\newpage
\section{Conclusion}
The study of political transitions in the Philippines and Indonesia shows that, despite different historical, cultural, and economic contexts, civil society played a crucial role in challenging powerful authoritarian regimes. These movements, often led by marginalized actors — students, NGOs, independent media, religious institutions — demonstrated a remarkable ability to rally diverse segments of the population around the common goal of political change. Their actions provided a powerful counterweight to state repression and allowed alternative visions of governance to emerge, even in environments where dissent had long been silenced.
However, this analysis also underlines an essential truth: while civil society can catalyze regime collapse, it alone does not guarantee the emergence of a stable and equitable democratic order. The fall of an authoritarian regime marks only the beginning of a much longer and more complex journey. Moving from protest to reconstruction requires strong and independent institutions, a culture of accountability, sustained political will, and a deep transformation of the mechanisms through which power is exercised and legitimized. The risk, otherwise, is to replace one form of domination with another, or to see former elites reinvent themselves under democratic façades. The return of figures linked to past authoritarian regimes — such as Ferdinand Marcos Jr. in the Philippines or Prabowo Subianto in Indonesia — is a stark reminder that the promises of democratic uprisings remain fragile. Their electoral success speaks to a broader phenomenon: the erosion of democratic memory, the persistence of clientelist networks, and the ability of old power structures to reassert themselves through populist rhetoric and strategic alliances. These developments should not be seen simply as setbacks, but as calls for civil society to renew its strategies, narratives, and modes of engagement. Civil society must therefore constantly reinvent itself to confront new forms of electoral authoritarianism and to defend — and deepen — the democratic gains of these transitions. This involves not only maintaining pressure on institutions, but also educating citizens, fostering political inclusion, and building coalitions across social divides. In many ways, the role of civil society is more demanding after the fall of an authoritarian regime than during the struggle itself, because it requires long-term commitment in the face of disillusionment, complexity, and the slow pace of reform. In response to the initial question, it appears that civil society, in both contexts, was indeed a key driver in the fall of authoritarian regimes. It was able to adapt to local constraints, build broad alliances, and embody the democratic aspirations of a growing portion of the population. Yet, the limitations observed after these transitions remind us that overthrowing a regime is not enough: institutionalizing change depends on civil society’s capacity to remain active, organized, and influential in the political arena over time. 



\end{document}