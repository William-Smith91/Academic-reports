\documentclass[12pt]{article}
\usepackage[utf8]{inputenc}
\usepackage[english]{babel}
\usepackage{amsmath, amssymb}
\usepackage{hyperref}
\usepackage{geometry}
\usepackage{titlesec}
\usepackage{setspace}
\usepackage{csquotes}
\usepackage{fancyhdr}
\usepackage{enumitem}
\usepackage{cite}

\geometry{margin=1in}
\setlength{\parindent}{0pt}
\setlength{\parskip}{1em}
\onehalfspacing

\title{Military Intervention as a Tool of Post-Soviet Russian Foreign Policy}
\author{William Evan Smith\\National Chengchi University}
\date{}

\pagestyle{fancy}
\fancyhf{}
\rhead{\thepage}
\lhead{Military Intervention as a Tool of Post-Soviet Russian Foreign Policy}

\begin{document}

\maketitle

\begin{abstract}
The 2022 invasion of Ukraine by Russia highlighted a central tool of Russian foreign policy: military intervention. This paper examines the reasons why post-Soviet Russia has resorted to armed interventions to achieve its objectives, culminating in a full-scale war against a sovereign state. The analysis begins with a definition of military intervention, exploring both its external and internal dimensions, as well as the political significance of the use of force. It then traces the evolution of this instrument within Russian foreign policy, from the presidency of Boris Yeltsin to the current regime of Vladimir Putin. By analyzing geopolitical contexts, the various forms of intervention, and the decision-making logics specific to each leader, this work aims to understand the growing role of military force in Russia’s international strategy. The study adopts a comparative methodology. It draws on a range of academic articles on Russian foreign policy, including case studies focused on the wars in Chechnya and Ukraine. These numerous case studies allow for an examination of the evolution of armed force usage. From peacekeeping missions in the 1990s to the invasion of Ukraine, this research demonstrates that military intervention has become an increasingly central tool of Russian power, and that the aggressiveness of such operations has intensified over the decades. This paper helps better understand how Russia’s changing use of military power reflects shifts in its geopolitical strategy and questions traditional views of the post-Cold War international order.
\end{abstract}
\newpage
\section*{Introduction}

Russia’s invasion of Ukraine, launched in February 2022, has served as a stark reminder to the international community that military force has once again become a central instrument of Moscow’s foreign policy. This large-scale military operation, carried out under the direct leadership of President Vladimir Putin, forms part of a broader trajectory of the gradual militarization of Russia’s external actions. Since the collapse of the Soviet Union in 1991, successive governments of the Russian Federation have resorted to various forms of military intervention to defend or promote their strategic interests, both at the regional and international levels.
In this context, military intervention—understood as the direct involvement of armed forces in a conflict or crisis beyond national borders, often for political, security, or geostrategic reasons—has become a structural component of Russian power. Whether in the form of peacekeeping missions, counterterrorism operations, annexations, or support for separatist entities, this practice has increased in frequency, diversified in nature, and intensified over the decades.
This study offers an in-depth analysis of the use of military intervention in the foreign policy of post-Soviet Russia, from the presidency of Boris Yeltsin to the contemporary era of Vladimir Putin, including the transitional period under Dmitry Medvedev. It therefore covers a period of over three decades, from 1991 to 2025, the year this paper was written.
The central question guiding this inquiry is the following: why has post-Soviet Russia so frequently resorted to the use of military force in its external relations? What are the objectives, underlying rationales, developments, and tangible outcomes of this strategy? The hypothesis put forward here is that the growing reliance on military violence follows a dual logic: on the one hand, the consolidation of power within an increasingly authoritarian regime, where military interventions serve to reinforce the legitimacy of the head of state; and on the other hand, a power strategy aimed at maintaining—or even restoring—Russia’s geopolitical influence on the global stage, particularly within the post-Soviet space.
To support this hypothesis, the study is organized around several key areas. First, it will situate military intervention within a theoretical framework by drawing on core concepts in international relations, notably hard power, regional security, and political legitimation. Next, a detailed analysis of several case studies—including the conflicts in Chechnya, the interventions in Georgia, Crimea, and the Donbas, as well as the full-scale invasion of Ukraine—will help identify the recurring patterns and developments in Russia’s military practice. Finally, a comparative approach will highlight the objectives pursued in each intervention, the results achieved, and the ways in which these experiences have influenced Moscow’s broader strategic posture.
Through this approach, the aim is not merely to recount events, but to understand the political, military, and ideological logics driving the militarization of Russian foreign policy and its implications for the contemporary international order.


% (You may continue the rest of the introduction here or directly jump into the next section as below)

\section*{Definition and Theory}

In the field of international relations, military intervention is fully embedded within what is referred to as hard power. This concept was theorized by the American political scientist Joseph Nye in his book Bound to Lead: The Changing Nature of American Power (1990), and was later developed further in subsequent publications. It remains a cornerstone of contemporary analysis concerning state strategies of power projection.
Hard power refers to a state's ability to coerce or compel other international actors to act in a certain way, primarily through the use of military force or the imposition of economic sanctions. It stands in contrast to soft power, also developed by Nye, which is based on non-coercive mechanisms of influence such as culture, political values, institutional attractiveness, or public diplomacy. While soft power seeks to persuade and attract, hard power aims to impose, deter, or neutralize through tangible and direct means.
Military intervention is thus one of the most explicit and visible manifestations of hard power. It involves the deliberate deployment of a state's armed forces on foreign territory with the objective of defending strategic interests, responding to perceived security threats, or altering the regional or international balance of power to its advantage. Such intervention can take various forms, including peacekeeping operations under the auspices of international organizations such as the UN or the CSTO; unilateral actions described as humanitarian or counterterrorism missions; targeted strikes; or full-scale conventional wars, sometimes involving prolonged occupations.
It is also important to distinguish between different types of military intervention based on either declared or underlying political intentions. Some aim to protect civilian populations or stabilize crisis zones, while others are driven by logics of territorial expansion, strategic dominance, or the reinforcement of influence in areas deemed vital. In the case of post-Soviet Russia, interventions often reflect a broader objective of maintaining or reestablishing a sphere of influence across the former USSR, a space commonly referred to as the "near abroad" (blizhnee zarubezh’ye).
Moreover, several theoretical approaches can help clarify the motivations and rationale behind the use of military force in international affairs. Classical realism, which asserts that states act primarily based on national interest and the pursuit of power, views military intervention as a legitimate and natural instrument for survival in an anarchic international system. Conversely, liberal theories place greater emphasis on the normative legitimacy of interventions, their legal frameworks, and compliance with international law.
Finally, from more critical or constructivist perspectives, some scholars argue that the use of force is not merely a matter of rational calculation, but also responds to symbolic and identity-based logics: asserting sovereignty, restoring national prestige, or challenging an international order perceived as unfair or threatening. This appears especially relevant in the case of contemporary Russia, where military interventions are often accompanied by a strong ideological narrative centered on the protection of Russian-speaking populations, the defense of historical memory, and resistance to perceived Western encirclement.
Therefore, to understand military intervention as a foreign policy instrument, one must position it at the intersection of multiple dynamics—strategic, political, and ideological—and analyze its concrete applications through empirical case studies, which this paper will explore in the following sections.


% Continue the theoretical section here.

\section*{Case Study}
Marked by the collapse of the Soviet Union on December 26, 1991, the former Soviet republics were shaken by successive crises throughout the 1990s. The turmoil sparked numerous conflicts in countries surrounding Russia, while separatist movements also emerged within the Russian Federation itself. The use of military force took many forms in the post-Soviet space—ranging from peacekeeping to outright land invasions. During Boris Yeltsin’s presidency (1991–1999), military intervention became both frequent and varied, serving multiple objectives in line with Russian foreign policy interests and the rapidly evolving geopolitical landscape.
\subsection*{Peacekeeping}

Peacekeeping refers to efforts, generally coordinated by international organizations like the United Nations, to help maintain or restore peace in conflict zones. Under Yeltsin, Russian peacekeeping missions began as early as 1992, in a context where Moscow was deeply concerned about the eruption of ethnic and national conflicts in newly independent states bordering Russia. Several key considerations motivated this involvement: in some of these regions, sizable Russian populations were perceived to be at risk and might require protection; the spillover effect of conflicts threatened to destabilize Russia’s own borders; the success of separatist groups abroad was feared to encourage similar secessionist tendencies within Russia’s diverse regions; the spread of Islamic fundamentalism, especially from Central Asia and the Caucasus, was viewed as a direct security threat to Russia’s southern flank; and lastly, Russia’s self-perception as a great power carried an implicit obligation to maintain regional stability and order.
Unlike UN-led operations, these interventions were often regional in scope and focused primarily on the post-Soviet sphere, with the explicit goal of preserving and projecting Russian influence in what Moscow considered its privileged “near abroad.” Following the dissolution of the USSR, several armed conflicts broke out across this space, prompting Russia to intervene militarily under the pretext of peacekeeping missions. This applied to five major conflicts: the Islamist uprising in Tajikistan (1992–1997), where Russia provided critical military support to the government against fundamentalist rebels; the two separatist conflicts in Georgia—in Abkhazia and South Ossetia—where Russian forces, officially deployed under the Commonwealth of Independent States (CIS) mandate, were frequently accused of tacitly supporting separatists; the Transnistria conflict in Moldova (1992), where Russian troops intervened militarily and remain stationed to this day as a frozen conflict; and finally, the Nagorno-Karabakh conflict between Armenia and Azerbaijan, in which Russia played a complicated dual role as both mediator and power broker, pursuing its own geopolitical interests while attempting to maintain a semblance of balance.
Although these deployments were formally presented as peacekeeping operations, Moscow’s neutrality was regularly called into question, revealing broader strategic objectives of maintaining dominance and influence over the former Soviet space (Donaldson & Nogee, 2014). The notion of the “near abroad” (blizhnee zarubezh’ye in Russian) became central to Russian foreign policy during this period—a concept that defined the newly independent non-Russian former Soviet republics as a vital sphere of political, economic, military, and cultural influence that Russia could neither relinquish nor afford to lose. This framework underpinned many of Yeltsin’s foreign policy decisions and set the stage for ongoing contestation in the region.


% Continue Peacekeeping, then...

\subsection*{The First Chechen War}
This conflict marked a significant turning point in the evolution of Russian military intervention. Beginning in late 1994, the operation’s primary objective was to reintegrate the Chechen Republic, which had embarked on a path toward de facto independence from the Russian Federation, and to reassert central government authority over a strategically important and volatile region. Military force, however, was not the government’s initial approach. Moscow first attempted to destabilize the separatist leadership by supporting rival Chechen opposition groups and imposed a harsh economic blockade that crippled the local economy and sought to undermine the separatist administration’s legitimacy.
The failure of these initial efforts eventually led President Yeltsin to authorize a full-scale military invasion, marking the beginning of a brutal and protracted conflict. The war was characterized by intense urban combat, guerrilla warfare tactics employed by Chechen fighters, and an often heavy-handed and indiscriminate response from Russian forces. The scale of the operation far exceeded localized skirmishes, evolving into a full-blown war with massive troop deployments, widespread destruction of towns and villages, and severe disruptions to civilian life. The conflict resulted in a high number of casualties on both sides, with civilian deaths and infrastructure damage mounting as the conflict dragged on.
Strategically, the First Chechen War illustrated the concrete application of classic military theory, particularly that of Carl von Clausewitz. Russian forces attempted a combination of maneuver warfare to isolate and encircle enemy fighters, attrition tactics aimed at exhausting their capacity to resist, and punitive operations targeting the civilian population designed to erode local support for the insurgency. Despite these efforts, Russia was unable to secure a decisive military victory. The war culminated in a significant defeat for the Kremlin in 1996, leading to a ceasefire and the granting of de facto autonomy to Chechnya under the Khasavyurt Accord.
The war had a lasting impact on Russian military doctrine and domestic politics. It reinforced a trend toward increasingly direct and forceful interventions in conflict zones perceived as critical to national security, a model that would be refined and replicated in subsequent decades. The conflict also left deep psychological and social scars within the North Caucasus region, triggering cycles of violence and radicalization that would eventually culminate in the Second Chechen War at the end of the 1990s.
In 1999, just two years after this costly defeat in Chechnya, President Yeltsin appointed a new prime minister who would profoundly reshape Russian foreign and military policy: Vladimir Putin, a former KGB officer whose tenure would define Russia’s approach to military intervention and geopolitical strategy for the following quarter-century.


\subsection*{The Putin Era}
Under Vladimir Putin’s various terms—including during Dmitry Medvedev’s interim presidency (2008–2012)—military intervention became a central and recurring feature of Russia’s foreign policy. This transformation was far from a temporary response; it represented a long-term strategic reorientation aimed at reclaiming what the Kremlin perceives as Russia’s rightful status on the global stage, particularly within the post-Soviet sphere. Over time, the deployment of armed force grew more frequent, calculated, and increasingly transparent, shedding earlier pretexts such as humanitarian intervention or peacekeeping missions. Alongside conventional military action, hybrid tactics—including cyber warfare, disinformation campaigns, and covert operations—also became integral to Moscow’s approach.
This shift was underpinned by a potent ideological framework that blended Soviet nostalgia, a firm rejection of Western unipolar dominance, and an assertive defense of Russian strategic interests in the face of NATO’s eastward expansion. Within this context, key military episodes—the Second Chechen War, the 2008 conflict with Georgia, the 2014 annexation of Crimea, and the 2022 full-scale invasion of Ukraine—are manifestations of a broader doctrine aimed at consolidating control over Russia’s near abroad and reasserting great power status in opposition to Western influence. Moreover, Russia’s encouragement of “frozen conflicts” in areas such as Transnistria, Abkhazia, and South Ossetia reflects a deliberate strategy to maintain leverage through regional instability.
Frequently described as imperialist or neo-imperialist by analysts, this strategic orientation rests on the conviction that military power remains indispensable to safeguarding sovereignty, shaping regional balances, and deterring neighboring countries’ political independence. Military intervention has thus become a core pillar of Russian foreign policy, deployed not only to contain crises but also to strengthen Moscow’s bargaining position internationally and bolster domestic legitimacy. This is further evidenced by sustained investment in military modernization and the expansion of Russia’s strategic capabilities.
In summary, the Putin and Medvedev years illustrate a merging of increasingly militarized foreign policy with authoritarian state consolidation, where military strength serves not only external objectives but also reinforces a nationalist narrative central to domestic political cohesion. The state’s promotion of patriotic education, large-scale military parades, and a selective reinterpretation of history reinforce this dynamic.


\subsection*{The Second Chechen War}
Shortly after Vladimir Putin assumed the role of prime minister, Russia launched a large-scale military campaign in Chechnya amid escalating instability. The immediate pretext was a series of devastating bombings in Russian cities during late 1999, which killed hundreds of civilians. While these attacks were officially attributed to Chechen separatists, lingering doubts persist regarding the precise origins and possible complicity of security forces. Regardless, the bombings provided the Kremlin with a powerful justification to initiate a forceful reconquest of the republic.
Unlike the earlier conflict, the Second Chechen War featured a more coordinated and ruthless strategy. The campaign was not merely a military operation but also a critical political tool for Putin, who rapidly gained popularity by presenting himself as the strong leader capable of restoring order after the turmoil of the 1990s. State-controlled media played a vital role in framing the conflict as a fight against terrorism and lawlessness.
The military campaign relied heavily on attritional tactics, including sustained aerial bombardments, encirclement of rebel positions, and harsh reprisals against suspected collaborators among the civilian population. Grozny, the Chechen capital, was subjected to extensive destruction and became a potent symbol of the conflict’s severity. Numerous human rights abuses, including violations of international humanitarian law, were documented by NGOs and international observers. The widespread use of “filtration camps” and other repressive measures deepened local grievances and complicated prospects for reconciliation.
Politically, the war was instrumental in enabling Putin to consolidate power, portraying himself as the decisive leader Russia needed. The nationalist overtones of the campaign, amplified by media propaganda, helped secure his election to the presidency in March 2000. It also led to the empowerment of security agencies like the FSB, reinforcing mechanisms of domestic control.
However, the aftermath was bleak. The war resulted in massive casualties, widespread destruction, and social fragmentation. A sustained campaign to root out insurgents persisted, while the conflict contributed to the emergence of Islamist militancy in the North Caucasus, which later expanded into Dagestan, Ingushetia, and beyond, presenting long-term challenges for Russian security policy.



\subsection*{2008 Russia-Georgia War}
The 2008 Russia-Georgia war marked a major turning point in Russian foreign policy, as it was the first direct invasion of a sovereign state internationally recognized since the end of the Cold War. This military intervention occurred in the context of Georgia’s growing rapprochement with NATO and the United States, which Moscow perceived as a direct strategic threat. The Russian operation, swift and decisive, aimed to assert control over the separatist regions of South Ossetia and Abkhazia while sending a clear signal to the West about the limits not to be crossed regarding influence in the post-Soviet space. Despite the severity of this aggression, international reactions, although condemning Russia’s actions, remained limited, illustrating the geopolitical complexity of the region and the great powers’ cautious stance toward Moscow.

\subsection*{Annexation of Crimea and Invasion of Ukraine}
The conflicts in Chechnya and Georgia gradually established military force as Moscow’s preferred tool to maintain and expand its influence over its neighbors. This dynamic continued and intensified with the Ukrainian crisis. Following the 2014 Maidan revolution, which pushed Ukraine toward a pro-Western orientation, Russia quickly responded by launching a large-scale military operation to seize the Crimean Peninsula (Götz & Ekman, 2023). This annexation was carried out covertly, using military units without clearly identifiable national insignia, nicknamed the “little green men.” At the same time, Moscow actively supported pro-Russian separatist movements in the Donbass region, sparking a prolonged armed conflict.
On February 24, 2022, the war escalated to a new level with Russia’s launch of a full-scale invasion of Ukraine, marking a major escalation in the conflict. As in the Chechen wars, Russia’s initial strategy relied on rapid maneuvers aimed at encircling and quickly defeating Ukrainian forces, but this phase ended in a notable failure. The war then turned into a fierce war of attrition, with Russian forces adopting punitive tactics intended to weaken Ukrainian resistance through massive bombardments and targeted attacks. However, despite these efforts, Moscow struggled to defeat a country both much larger and better supported internationally, facing robust Ukrainian resistance and growing Western support (Iliyasov & Herrera, 2022).


\section*{Objectives and Results}

The various case studies analyzed show that since the establishment of the Russian Federation, military interventions have been numerous, diverse in form, and driven by foreign policy objectives that vary according to geopolitical context, historical period, and the leadership in place. These interventions reflect both a reaction to external developments and a deliberate strategy designed to consolidate Russia’s influence regionally and globally.
Under the presidency of Boris Yeltsin, military interventions often took the form of peacekeeping missions, particularly in the former Soviet republics. Officially presented as initiatives aimed at stabilizing conflict zones, these missions did help reduce certain tensions, as seen in Georgia and Moldova during the 1990s. However, beyond their declared goal of peacekeeping, these operations also served more political and strategic objectives, including the indirect support of pro-Russian separatist movements opposed to central governments seeking closer ties with the West or asserting full national sovereignty. This dual-purpose approach—balancing conflict containment with influence projection—often resulted in "frozen conflicts", where no lasting resolution was achieved, yet Russia succeeded in maintaining a form of de facto control over key areas and political processes in those regions (Donaldson & Nogee, 2014). These conflicts also created zones of instability that allowed Moscow to position itself as an indispensable actor in any future negotiation processes.
The First Chechen War (1994–1996) reflects a markedly different logic of intervention: a direct and unilateral military attempt to reintegrate a republic that had declared de facto independence. While the primary aim was to restore territorial integrity, the war also became a domestic political maneuver intended to reassert the authority of the Russian state and salvage Yeltsin’s declining political standing, which had been severely weakened after the 1993 constitutional crisis and growing public disillusionment with post-Soviet reforms. The campaign was launched with limited preparation and underestimated the resolve and capabilities of Chechen fighters. Ultimately, the war ended in both political and military failure, with Russian forces withdrawing under pressure and the Chechen opposition retaining control of the region.
The Second Chechen War, initiated in 1999, pursued similar goals but was framed in a new political context. Now led by Vladimir Putin, the intervention also served a clear leadership-building function, helping to forge his public image as a decisive and uncompromising leader. The war was significantly more intense, characterized by extensive bombardments, harsh counter-insurgency tactics, and a highly centralized command structure. It resulted in Moscow re-establishing control over the region, but at a tremendous human and infrastructural cost. Moreover, the framing of the conflict within the global war on terror, especially after the September 11, 2001 attacks, allowed Russia to justify ongoing military operations in the North Caucasus as part of a broader counterterrorism effort. This gave the Kremlin greater latitude in its domestic and foreign narratives, aligning its actions with global security discourses while consolidating internal power.
The 2008 war in Georgia marked a clear departure from past interventions. For the first time, Russia conducted direct military operations against a sovereign state with international recognition, outside the borders of the Russian Federation. Ostensibly launched to protect Abkhazia and South Ossetia from Georgian aggression, the operation also aimed to halt Georgia’s Euro-Atlantic integration and to signal to other post-Soviet states the consequences of moving too close to the West. The intervention, though short, was decisive and demonstrated Moscow’s readiness to use force to enforce its red lines in what it considers its strategic backyard. It also underscored the shift from defensive to preemptive and coercive uses of military power in Russian foreign policy.
This trend continued and deepened with the annexation of Crimea in 2014 and the initiation of conflict in the Donbas region. Moscow presented its actions as a response to the Maidan revolution and alleged threats against Russian-speaking populations. In reality, the move was a calculated effort to undermine Ukraine’s sovereignty, solidify control over the Black Sea, and send a message to NATO regarding the limits of its eastern expansion. The conflict introduced a hybrid warfare model that combined military, informational, and covert tactics, blurring the lines between formal war and political destabilization. With Crimea’s annexation, Russia signaled that it was prepared to redraw international borders unilaterally and defy the post–Cold War order.
The full-scale invasion of Ukraine on February 24, 2022, marked the culmination of this militarized trajectory. Framed by the Kremlin as a “special military operation” to “denazify” and “demilitarize” Ukraine, the invasion concealed much broader aims: the toppling of the Kyiv government, reversal of Ukraine’s westward orientation, and potentially the reconfiguration of Eastern Europe’s geopolitical landscape. Unlike earlier interventions, this campaign encountered unprecedented resistance—both militarily from Ukraine and diplomatically from a globally united front. The initial blitz strategy aimed at a rapid decapitation of Ukrainian leadership failed, forcing Russia to shift toward a long-term war of attrition. This involved punishing strikes on civilian infrastructure, attempts to break morale, and sustained fighting in key regions. However, the war has revealed the limits of Russian military power, especially in the face of organized resistance, modern Western weaponry, and severe international sanctions.
Ultimately, Russia’s military interventions follow an adaptive and increasingly assertive logic, where force is no longer used solely for defense or crisis management but is now an integrated instrument of statecraft. These operations serve a dual purpose: to stabilize the regime internally through narratives of strength and protection, and to expand Russia’s influence regionally through coercive means. Whether aiming to secure territorial claims, deter Western alignment, or project symbolic dominance, military force is now deeply embedded in the strategic DNA of post-Soviet Russian foreign policy.


% Continue the analysis and conclusion.
\newpage
\section*{References}
\begin{itemize}[leftmargin=*, label={}]
    \item Götz, E., \& Ekman, P. (2023). \textit{Russia’s War Against Ukraine: Context, Causes, and Consequences.} European Security, 32(1), 1–19. \url{https://doi.org/10.1080/09662839.2023.2172814}
    \item Donaldson, R. H., Nogee, J. L., \& Nadkarni, V. (2022). \textit{The Foreign Policy of Russia: Changing Systems, Enduring Interests} (6th ed.). Routledge.
    \item Iliyasov, M., \& Herrera, Y. M. (2022). \textit{Russia’s war strategy: What Chechnya suggests for Ukraine.} Journal of Strategic Studies, 45(4), 567–589. \url{https://doi.org/10.1080/01402390.2022.2064287}
\end{itemize}

\end{document}