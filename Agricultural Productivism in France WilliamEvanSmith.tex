\documentclass[12pt]{article}
\usepackage[utf8]{inputenc}
\usepackage[a4paper,margin=2.5cm]{geometry}
\usepackage{setspace}
\usepackage{hyperref}
\usepackage{lmodern}
\usepackage{parskip}
\usepackage{titlesec}
\usepackage{enumitem}
\usepackage{csquotes}

\title{The Effects of Agricultural Productivism in France}
\author{William Smith \\ A2DD – Sciences Po Saint Germain en Laye}
\date{}

\begin{document}
\maketitle
\onehalfspacing
\noindent





The purpose of this text is to explore the socio-economic effects of agricultural productivism in France. The work will focus on a historical perspective and certain points from a sociological angle. The paper aims to address various questions about the socio-economic effects of agricultural productivism, drawing on different works of history and sociology on the subject of agricultural productivism in France.

The paper is intended to answer various questions raised by the topic. How has productivism impacted the French rural world ? How has the work of farmers evolved during this period ? Why was a new economic model for French agriculture created ?

The paper is structured in several parts. First, the concept of agricultural productivism will be defined and explained, followed by an examination of the form of agriculture and the French rural world before the advent of agricultural productivism. Subsequently, the transformation of French rural life and its economy through the application of productivism by various actors constituting the productive agricultural economy will be discussed. Finally, the long-term consequences of the radical change in agricultural production in French society will be examined.

The term "productivism" is a neologism used to describe the development process followed, in our case, by French agriculture since the late 1950s (Jean-Claude Tirel). It involves systematic efforts to improve or increase agricultural productivity through the increased use of phytosanitary products and mechanization to maximize crop and livestock yields. This agricultural approach developed widely after World War II in the United States and Western Europe. Agricultural productivism, sometimes also called intensive agriculture, facilitated the emergence of a consumer society in the West. Productivism is often described by its proponents as the modernization of agriculture, replacing traditional agricultural practices where the figure of the peasant is considered outside of "modernity" and destined to disappear, replaced by agricultural entrepreneurs (Estelle Deléage, 2012).

Before the advent of agricultural productivism in France, peasants, a term dating back to the 11th century meaning inhabitants of a country who cultivate the land, formed a significant layer of French society. According to the article "Les révolutions vertes de la campagne française," the active peasant population represented a quarter of the workforce in France in 1954. Peasants produced little, and it is estimated that half of agricultural production remained on the farm on the eve of World War II.

The world of peasant agriculture relied on countless small agricultural operations where life remained local, and agricultural cooperatives remained at the village level. Peasants lived apart from the general economy, which more often affected professions in urban areas. The farms were often family-owned, with strong social reproduction among the French peasantry – one was born a peasant. They relied much more on the work of the land than on the commercialization of their strict production, making it difficult for both peasants and policymakers to assimilate family farming as a business. The peasant agricultural production method was seen more as a way of life than a simple agricultural practice (André Tel, 1985).

If the Industrial Revolution had an impact on the French rural world in the 19th century, causing a significant rural exodus, a large number of French people were still farmers in the 1950s when various productivist practices were put in place, greatly changing the rural economy of Hexagon.

French agriculture underwent a radical transformation after World War II. The figure of the peasant in the post-war period was largely associated with the Peasants' Corporation, an agricultural union created by the Vichy regime (Estelle Deléage, 2012). Peasantry was seen as an essential element to enable the National Revolution, a political project of Philippe Pétain. After 1945, France was a country that was not self-sufficient in terms of food. The economic and political context led to the modernization of French agriculture from a more traditional model to a productivist model, where the peasant was replaced by the modern farmer, sometimes also called an agricultural entrepreneur.

The modernization of French agriculture was driven by professional agricultural unions such as the National Center of Young Farmers (CNJA), formed within the Catholic Agricultural Youth (JAC). The goal of young farmers was to emancipate the lives of peasants through various processes: concentration and enlargement of farms, specialization of livestock and crops, maximization of agricultural production (André Tel 1985, Estelle Deléage 2012). The productivist project was taken up by the National Federation of Farmers' Unions (FNSEA), the majority agricultural union co-managing with the state to consolidate small farms below the profitability threshold and assist voluntary departures of peasants with compensation. Small farms that produced little but were numerous were replaced by larger farms with much fewer workers than in the traditional agricultural model.

Agricultural productivism is achieved through a change in the model of farms and the use of mechanized equipment and phytosanitary products to improve production. The agricultural modernization advocated by the CNJA involves maximizing production and the development of mechanized equipment, fertilizers, and insecticides, allowing the increase in production desired by agricultural unions. Phytosanitary products are a real issue in the productivist model.

According to the document "Images of Agricultural Productivism: A Sociohistorical Analysis of Technical Guide Covers (1959-2014)," through the study of images produced by agricultural cooperatives, certain trends can be observed, including one between the 1960s and 1970s, where the promotion of chemical fertilizers and pesticides is very strong among farmers who need convincing. Some of them experienced an agricultural model where the use of chemical products was almost nonexistent. The boom in agrochemistry developed during these years, with agricultural cooperative brochures such as "Providence Agricole" highlighting the use of chemical treatments for crops and pesticide spraying. These practices are seen by these actors as the medicalization of crops. Criticisms of this model appear as early as the 1970s, with the increase in farmers' indebtedness and environmental degradation (Robin Villemaine, 2017).

Agricultural productivism is thus advocated by farmers and agricultural union organizations, which have often been staunch supporters of technological progress, here referring to the use of new technical or agrochemical means to improve agricultural production at any cost. The pursuit of agricultural abundance through the industrialization of agriculture is the goal of the actors who have implemented the productivist model in France.

The consequences of productivist models have led to major changes in the French rural world and the agricultural sector in particular. Agricultural productivism since the "Trente Glorieuses" has led to the decline in the number of farmers among the active population in France, even in rural areas, with 1.7\textbackslash{}% of the active population in metropolitan France and 5.5\textbackslash{}% in rural areas in 2009 (Gilles Laferté, 2022). While agriculture before World War II was considered separate from the economy, the modernization of agriculture made it much more connected to the national and international economy than some sectors. Many functions performed by farmers have gradually been taken over outside the farm by external actors. The work of the peasant is subject to agro-supply industries that provide seeds, fertilizers, phytosanitary products, and agri-food industries that buy agricultural production for processing (Jean-Claude Tirel, 1983). The value of agricultural products is more under the control of the buyer than the farmer.

The traditional farmer, the peasant before World War II, lived off the work of the land. In the productivist model, the agricultural operator lives off the trade of their agricultural production. While a portion of the production remained on the farm before, the export of French agricultural products to third countries became an objective for advocates of agricultural productivism. The specialization of crops between French regions led to the abandonment of certain agricultural regions deemed less profitable, while France imported many agricultural products, as well as machinery and agricultural chemicals, to maximize the harvests of its own farmers (Jean-Claude Tirel, 1983).

\newpage
\section*{Conclusion}

Agricultural productivism has brought about significant socio-economic changes in France, especially in rural areas. Peasants, who formed traditional agriculture characterized by small but numerous farms, were transformed into agricultural operators, agricultural entrepreneurs, after World War II through the modernization of agriculture ensured by agricultural unions and supported by the state. The productivist model caused a strong concentration and specialization of agricultural operations, a reduction in the number of farmers, the significant use of mechanized equipment and phytosanitary products for better agricultural yields. Agriculture integrated into the globalized economy and became dependent on actors who did not play a significant role in the traditional agricultural model but played a crucial role in the productivist model—agri-food and agri-supply industries.

The document, focusing on socio-economic effects, briefly describes one of the most obvious changes brought about by agricultural productivism in France—its impact on the environment, an important issue given that the use of chemical fertilizers and pesticides causes soil contamination and negatively impacts biodiversity. The impacts of agricultural productivism affect the very perpetuation of the model itself, with soil degradation affecting productivity. The fact that the renewal of retiring farmers will pose new challenges to the continuity of a model that has sought to reduce their number for a long time generates new developments in the French agriculture sector.
\newpage
\section*{References}

DELÉAGE, Estelle. (2012). Les paysans dans la modernité. Revue Française de Socio-Économie. 2012. Vol. 9, n° 1, pp. 117‑131. DOI 10.3917/rfse.009.0117.

FEL, André. (1985). Les révolutions vertes de la campagne française (1955-1985). Vingtième Siècle. Revue d’histoire. 1985. Vol. 8, n° 1, pp. 3‑18. DOI 10.3406/xxs.1985.1200. Distributor: Persée - Portail des revues scientifiques en SHS publisher: Centre National du Livre.

LAFERTÉ, Gilles. (2022). Capitalisme agricole et normalisation sociale : les agriculteurs au contact de la ville. Métropolitiques [en ligne]. 31 octobre 2022. Disponible à l’adresse : https://metropolitiques.eu/Capitalisme-agricole-et-normalisation-sociale-les-agriculteurs-au-contact-de-la.html [Consulté le 23 décembre 2023].

TIREL, Jean-Claude. (1983). Le débat sur le productivisme. Économie rurale. 1983. Vol. 155, n° 1, pp. 23‑30. DOI 10.3406/ecoru.1983.2961. Distributor: Persée - Portail des revues scientifiques en SHS publisher: Société Française d’Economie Rurale.

VILLEMAINE, Robin. (2017). Le productivisme agricole en images. Une analyse sociohistorique de couvertures illustrées de guides techniques (1959-2014). Images du travail, travail des images [en ligne]. 1 septembre 2017. N° 4. DOI 10.4000/itti.1017. [Consulté le 23 décembre 2023].
\end{document}